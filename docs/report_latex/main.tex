\documentclass[12pt,a4paper]{apm_report}

\graphicspath{{images/}}

% Report metadata
\reporttitle{Plateforme d'observabilite APM}
\reportsubtitle{Supervision applicative et performance des donnees}
\reportauthor{Nawfal RAZOUK \\ Mathilde Soulet \\ Avel LE MEE MORET}
\reportsupervisor{Mouad RIYAD}
\reportinstitute{ENSMR}
\reportdegree{3A Genie Informatique - Option : IData}
\reporttype{Rapport du projet}
\reportlocation{Rabat}
\reportdate{\today}
\reportproject{APM-Observability}
\reportcourse{Administration de donnees}
\reporttopic{TimescaleDB}
\reportlogo{images/logo.png}


\begin{document}
\maketitle

\begin{center}
{\Large\bfseries Résumé}
\end{center}
Ce rapport s’inscrit dans le cadre du module Administration de données et décrit la conception et la mise en œuvre d’APM-Observability, une plateforme d’observabilité applicative. Après une synthèse sur TimescaleDB et le modèle des séries temporelles, nous présentons l’architecture globale : pipeline d’ingestion via API Django, stockage en hypertables, vues continues pour les KPI et exposition des métriques vers Prometheus. La partie pratique détaille la configuration en mode cluster, la génération de données, les sauvegardes et restaurations pgBackRest vers MinIO, ainsi que la validation par tests Postman/Newman. Enfin, la supervision est illustrée par des tableaux de bord Grafana et des indicateurs de performance (latence, taux d’erreur, volumétrie). L’ensemble fournit une base opérationnelle pour le suivi applicatif et la prise de décision.

\begin{center}
{\Large\bfseries Abstract}
\end{center}
This report, developed for the Data Administration course, presents the design and implementation of APM-Observability, an application performance monitoring platform. We describe the data pipeline from ingestion to storage in TimescaleDB hypertables, the use of continuous aggregates for KPI computation, and the export of metrics to Prometheus. Operational aspects include cluster deployment, data seeding, pgBackRest backups to MinIO, and automated validation with Postman/Newman. The monitoring layer is showcased through Grafana dashboards covering availability, latency, error rate, and system health. The project delivers a practical, reproducible foundation for observability in data-intensive applications.

\clearpage
\tableofcontents
\clearpage
\listoffigures
\clearpage

\section{Introduction}

L’observabilité applicative (APM) vise à comprendre le comportement d’un système à partir de ses signaux (métriques, traces et logs). L’objectif de ce rapport est de présenter une plateforme d’observabilité basée sur TimescaleDB pour le stockage des données temporelles et Grafana pour la visualisation.

Nous décrivons d’abord les concepts et particularités de la base TimescaleDB, puis la mise en œuvre technique du projet. Nous présentons ensuite les usages pratiques (sauvegarde, restauration, seeding et optimisation) avant d’aborder la supervision avec Grafana. Enfin, une conclusion synthétise les choix et propose des perspectives.

Le code source du projet est disponible sur GitHub :
\par\noindent\url{https://github.com/NawfalRAZOUK7/apm-observability}

\clearpage
\section{Présentation de la base}

TimescaleDB est une base de données open-source spécialisée dans la gestion des séries temporelles. Elle se présente sous la forme d’une extension de PostgreSQL, ce qui lui permet de bénéficier de la fiabilité et de l’écosystème PostgreSQL, tout en ajoutant des optimisations dédiées aux données horodatées (\textit{time-series}).~\cite{timescaledb-docs}

Elle est conçue pour gérer efficacement de très grands volumes de données insérées en continu, comme les métriques de supervision, les données IoT (Internet of Things), les logs applicatifs ou les données financières.

\begin{figure}[H]
    \centering
    \includegraphics[width=1\textwidth]{timescaledb.png}
    \caption{Page d'accueil du site de TimescaleDB}
\end{figure}

\subsection{Use Case de la base}

TimescaleDB est utilisée par de nombreuses entreprises à l’échelle mondiale, notamment dans les domaines du monitoring, de l’IoT et de la finance.~\cite{timescaledb-customers}

\subsubsection*{Exemples d’entreprises utilisatrices}
\begin{itemize}
    \item Comcast : supervision de réseaux et métriques de performance
    \item Cisco : monitoring d’équipements réseau
    \item IBM : gestion de données IoT
    \item Schneider Electric : capteurs industriels et énergie
    \item T-Mobile : analyse des performances réseau
\end{itemize}

\subsubsection*{Cas d’usage typiques}
\begin{itemize}
    \item Monitoring d’infrastructure et observabilité
    \item Internet des objets (IoT)
    \item Analyse financière et trading
    \item DevOps (Development and Operations) et métriques applicatives
\end{itemize}

\subsection{Architecture interne de la base}

TimescaleDB repose sur l’architecture interne de PostgreSQL et ajoute une couche spécifique pour les séries temporelles.

\subsubsection*{Fichiers et stockage}
Les données sont stockées dans des \textit{hypertables}, qui sont automatiquement découpées en \textit{chunks} (partitions temporelles). Chaque chunk est physiquement une table PostgreSQL classique stockée sur le système de fichiers.

\subsubsection*{Processus internes}
TimescaleDB utilise des processus d’arrière-plan (\textit{background workers}) pour :
\begin{itemize}
    \item la compression automatique des données anciennes
    \item la suppression des données selon des politiques de rétention
    \item la mise à jour des agrégations continues
\end{itemize}

L’ensemble reste conforme aux propriétés ACID (Atomicité, Cohérence, Isolation, Durabilité) grâce au moteur PostgreSQL, basé notamment sur le WAL (Write-Ahead Logging) et le MVCC (Multi-Version Concurrency Control).

\subsection{Particularités de la base de données}

La principale particularité de TimescaleDB est le partitionnement automatique des données temporelles via les hypertables, totalement transparent pour l’utilisateur.

Elle propose également :
\begin{itemize}
    \item une compression columnaire très efficace (jusqu’à 90--95 \%)~\cite{timescaledb-compression}
    \item des fonctions temporelles avancées (ex. \textit{time\_bucket})
    \item des agrégations continues pour accélérer les requêtes analytiques
\end{itemize}

\subsection{Comparaison avec Oracle Database}

\begin{table}[H]
\centering
\begin{tabular}{lcc}
\toprule
\textbf{Critère} & \textbf{TimescaleDB} & \textbf{Oracle Database} \\
\midrule
Licence & Open-source & Propriétaire \\
Spécialisation & Séries temporelles & Généraliste \\
Partitionnement temporel & Automatique & Manuel (option) \\
Compression & Native & Option payante \\
Coût & Faible & Très élevé \\
\bottomrule
\end{tabular}
\end{table}

Oracle est une base de données généraliste adaptée aux charges OLTP (Online Transaction Processing) et OLAP (Online Analytical Processing), tandis que TimescaleDB est optimisée pour les séries temporelles avec un coût et une complexité bien moindres.

\subsection{Comparaison On-Premise et Cloud}

\subsubsection*{Coûts}
Une solution on-premise implique des coûts initiaux élevés (matériel, administration, maintenance). Les solutions cloud réduisent ces coûts grâce à un modèle à l’usage.

\subsubsection*{Performances}
\begin{itemize}
    \item On-premise : performances maximales si l’infrastructure est bien dimensionnée
    \item Cloud : légère latence réseau, mais excellente scalabilité
    \item Amazon Aurora (service cloud PostgreSQL managé) : bonnes performances mais coût plus élevé à grande échelle
\end{itemize}

\clearpage
\section{Mise en œuvre et expérimentation}

\subsection{Démarrage et connexion}

\subsubsection*{Lancement d'une instance TimescaleDB}
Le projet fournit un environnement Docker prêt à l’emploi pour lancer une instance TimescaleDB. Il suffit de copier le fichier d’exemple `.env` et de démarrer les services via Docker Compose :

\begin{verbatim}
cp .env.example .env
docker compose -f docker/docker-compose.yml up --build
\end{verbatim}

Cette commande initialise à la fois le conteneur TimescaleDB et le backend Django.

\subsubsection*{Connexion à la base de données}
Une fois le conteneur en fonctionnement, la connexion peut se faire directement depuis Docker :

\begin{verbatim}
docker compose -f docker/docker-compose.yml exec db psql -U apm -d apm
\end{verbatim}

Il est également possible de se connecter depuis l’hôte si le port est exposé :

\begin{verbatim}
psql -h localhost -p 5432 -U apm -d apm
\end{verbatim}

\subsection{Prise en main}

\subsubsection*{Création des tables}
Les modèles Django sont déjà définis. La commande de migration crée automatiquement les tables nécessaires, incluant les hypertables de TimescaleDB :

\begin{verbatim}
python manage.py migrate
\end{verbatim}

\subsubsection*{Insertion et manipulation des données}
Les données peuvent être insérées via l’API REST :

\begin{verbatim}
curl -X POST [http://127.0.0.1:8000/api/requests/](http://127.0.0.1:8000/api/requests/) 
-H "Content-Type: application/json" 
-d '{
"time": "2025-12-14T12:00:00Z",
"service": "billing",
"endpoint": "/health",
"method": "GET",
"status_code": 200,
"latency_ms": 42
}'
\end{verbatim}

Les requêtes peuvent ensuite être consultées avec :

\begin{verbatim}
curl [http://127.0.0.1:8000/api/requests/](http://127.0.0.1:8000/api/requests/)
\end{verbatim}

\subsubsection*{Fonctionnalités spécifiques de TimescaleDB}
Le projet utilise les particularités de TimescaleDB pour :

\begin{itemize}
    \item Créer des hypertables pour le partitionnement automatique.
    \item Mettre en place des agrégats continus (hourly/daily) pour accélérer les requêtes analytiques.
    \item Assurer un fallback sur PostgreSQL standard si TimescaleDB n’est pas disponible.
\end{itemize}

\subsection{Déploiement en cluster}

\subsubsection*{Architecture générale}
Le cluster repose sur plusieurs nœuds TimescaleDB orchestrés via Docker Compose, avec un backend Django commun pour l’API et l’ingestion des données.

\subsubsection*{Cluster TimescaleDB avec Docker Compose}
Chaque nœud est défini dans le fichier `docker-compose.yml`. Les conteneurs communiquent entre eux et partagent les configurations de réplication.

\subsubsection*{Communication entre les nœuds}
Les nœuds utilisent les ports internes Docker pour échanger les données et maintenir la cohérence des hypertables. Les services Django et TimescaleDB se connectent via des noms de service Docker.

\subsubsection*{Sécurisation des accès}
L’accès aux conteneurs se fait uniquement via les variables d’environnement définies dans `.env`. Les scripts de backup/restauration gèrent automatiquement les clés SSH pour sécuriser les opérations sans intervention manuelle.


\clearpage
\section{Application pratique}

\subsection{Sauvegardes et restauration}

\subsubsection*{Backups hot et cold}
Les sauvegardes sont assurées via pgBackRest avec deux dépôts (hot/cold) sur un stockage compatible S3 (MinIO). Les opérations sont automatisées et idempotentes.

\textbf{Stack backup dédiée} :
\begin{verbatim}
docker compose -f docker/docker-compose.backup.yml up -d --build
make setup-backup-ssh
docker compose -f docker/docker-compose.backup.yml exec pgbackrest \
  pgbackrest --stanza=apm info
docker compose -f docker/docker-compose.backup.yml exec pgbackrest \
  pgbackrest --stanza=apm --type=full backup
\end{verbatim}

\textbf{Mode cluster} :
\begin{verbatim}
make pgbackrest-full
make pgbackrest-full-repo2
\end{verbatim}

\begin{figure}[H]
    \centering
    \includegraphics[width=0.9\textwidth]{backup_flow.png}
    \caption{Flux de sauvegarde pgBackRest vers MinIO}
\end{figure}

\subsubsection*{Stockage compatible S3}
Les backups peuvent être exportés vers un stockage compatible S3 (MinIO). La configuration S3 se trouve dans :
\begin{itemize}
    \item \path{docker/backup/pgbackrest-client.conf}
    \item \path{docker/backup/pgbackrest-server.conf}
\end{itemize}
Les buckets hot/cold sont séparés pour isoler les stratégies de rétention.

Les Figures~\ref{fig:minio-hot} et~\ref{fig:minio-cold} illustrent la séparation hot/cold côté MinIO.
\begin{figure}[H]
    \centering
    \includegraphics[width=1\textwidth]{minio_pgbackrest.png}
    \caption{Bucket hot \texttt{pgbackrest}}
    \label{fig:minio-hot}
\end{figure}

\begin{figure}[H]
    \centering
    \includegraphics[width=1\textwidth]{minio_pgbackrest_cold.png}
    \caption{Bucket cold \texttt{pgbackrest-cold}}
    \label{fig:minio-cold}
\end{figure}

\subsection{Données de test}

\subsubsection*{Seeding de la base avec Faker}
Pour les tests et démonstrations, la base peut être peuplée avec des données factices générées via la librairie Faker.

\textbf{Mode cluster} :
\begin{verbatim}
make seed
\end{verbatim}

\textbf{Stack principale (local)} :
\begin{verbatim}
docker compose -f docker/docker-compose.yml exec web \
  python manage.py seed_apirequests --count 1000 --days 1
\end{verbatim}

\textbf{Script utilitaire} :
\begin{verbatim}
scripts/seed_faker.sh --count 1000 --days 1
\end{verbatim}

\subsection{Tolérance aux pannes}

\subsubsection*{Simulation de pannes}
Des tests de tolérance sont possibles en simulant l’arrêt de nœuds Docker pour vérifier la résilience du cluster. Des scripts de drill sont fournis dans \texttt{scripts/drills} :

\begin{verbatim}
CONFIRM=YES scripts/drills/02_failover_replica.sh
CONFIRM=YES scripts/drills/03_minio_outage.sh
\end{verbatim}

\subsubsection*{Procédures de recovery}
La restauration automatique depuis les backups permet de remettre rapidement le système en état de fonctionnement.

\begin{verbatim}
docker compose -f docker/docker-compose.backup.yml exec pgbackrest \
  pgbackrest --stanza=apm restore
CONFIRM=YES scripts/drills/01_primary_restore.sh
\end{verbatim}

\subsection{Optimisation}

\subsubsection*{Index et performances}
L’utilisation des hypertables, des agrégats continus et d’index spécifiques permet d’optimiser les performances des requêtes analytiques, notamment pour les KPI et les tableaux de bord. Après un gros seeding, un rafraîchissement des CAGG garantit des résultats cohérents :

\begin{verbatim}
python manage.py refresh_apirequest_hourly
python manage.py refresh_apirequest_daily
\end{verbatim}

\subsection{Tests et validation}
Les scénarios API sont automatisés via Postman et exécutés en CLI avec Newman. Les collections et environnements sont disponibles dans \path{postman/} :
\begin{itemize}
    \item \path{APM_Observability_Step1.postman_collection.json} \\
    \path{APM_Observability_Step5.postman_collection.json}
    \item \path{APM_Observability.cluster.postman_environment.json}
    \item \path{APM_Observability.main.postman_environment.json}
\end{itemize}

Les rapports HTML/JUnit de l’exécution sont disponibles dans \path{reports/all_tests_20251225_134101/}. Le tableau ci-dessous synthétise les résultats.

\begin{table}[H]
\centering
\begin{tabular}{lcccc}
\toprule
\textbf{Step} & \textbf{Tests} & \textbf{Failures} & \textbf{Errors} & \textbf{Time (s)} \\
\midrule
Step 1 & 14 & 0 & 0 & 3.394 \\
Step 2 & 4 & 0 & 0 & 1.232 \\
Step 3 & 3 & 0 & 0 & 0.285 \\
Step 4 & 2 & 0 & 0 & 0.410 \\
Step 5 & 5 & 0 & 0 & 0.609 \\
\midrule
\textbf{Total} & \textbf{28} & \textbf{0} & \textbf{0} & \textbf{5.930} \\
\bottomrule
\end{tabular}
\caption{Synthèse des tests Newman (Postman)}
\end{table}

\begin{figure}[H]
    \centering
    \includegraphics[width=1\textwidth]{tests_step1_report.png}
    \caption{Rapport Newman - Step 1 (CRUD + filtres)}
    \label{fig:tests-step1}
\end{figure}

\begin{figure}[H]
    \centering
    \includegraphics[width=1\textwidth]{tests_step5_report.png}
    \caption{Rapport Newman - Step 5 (KPIs / analytics)}
    \label{fig:tests-step5}
\end{figure}

\clearpage
\section{Supervision avec Grafana}

\subsection{Présentation de Grafana}
Grafana est une plateforme open-source de visualisation et de surveillance de données. Elle permet de créer des tableaux de bord interactifs et personnalisables, facilitant le suivi en temps réel des métriques provenant de diverses sources. Son interface intuitive rend l’analyse des données accessible, même pour des utilisateurs non techniques.

\subsection{Intégration avec TimescaleDB}
Grafana s’intègre facilement avec TimescaleDB, une base de données optimisée pour les séries temporelles. Cette intégration permet de récupérer et d’afficher efficacement des données historiques et en temps réel, offrant ainsi une visibilité complète sur les tendances et les anomalies.

\subsection{Dashboard de monitoring}
Le tableau de bord ci-dessous montre la supervision d’un système avec Grafana et TimescaleDB.

Métriques affichées :
\begin{itemize}
    \item Nombre de requêtes par heure (hits/hour)
    \item Latence moyenne par service
    \item Taux d’erreur par endpoint
\end{itemize}

\begin{figure}[H]
    \centering
    \includegraphics[width=1\textwidth]{images/grafana_dashboard.png}
    \caption{Dashboard Grafana de supervision}
\end{figure}

\subsection*{Conclusion}
Pour la majorité des cas d’usage modernes, une solution cloud offre le meilleur compromis entre coût, performance et simplicité d’exploitation, tandis que l’on-premise reste réservé aux contraintes spécifiques (sécurité, conformité, souveraineté des données).


\clearpage
\section{Conclusion et perspectives}

Ce travail a permis de mettre en place une plateforme d’observabilité APM cohérente, couvrant l’ingestion, le stockage, l’analyse et la visualisation des données. L’usage de TimescaleDB pour les séries temporelles, associé à des vues continues pour les KPI, apporte un socle performant et maintenable. La supervision via Prometheus et Grafana fournit une vision claire de la disponibilité, de la latence et du taux d’erreur, tandis que l’intégration des sauvegardes pgBackRest avec MinIO améliore la résilience globale. Les jeux de tests Postman/Newman et les scripts de déploiement documentés rendent la solution reproductible.

Sur le plan d’architecture, nous avons validé une configuration en mode cluster adaptée aux besoins de supervision et de charge. Ce choix facilite l’évolutivité, la séparation des responsabilités et la montée en puissance future. Dans un contexte de production, le cloud reste souvent le meilleur compromis en termes de coût, d’élasticité et d’exploitation, tandis que l’on-premise répond surtout aux exigences fortes de souveraineté et de conformité.

\subsection*{Perspectives}
Plusieurs axes peuvent enrichir la plateforme :
\begin{itemize}
    \item mise en place d’alertes avancées et d’objectifs SLO/SLA ;
    \item ajout de traces distribuées (OpenTelemetry) pour compléter métriques et logs ;
    \item optimisation des coûts via politiques de rétention, compression et archivage ;
    \item industrialisation CI/CD pour automatiser tests, migrations et déploiements ;
    \item durcissement sécurité (RBAC, rotation des secrets, auditabilité).
\end{itemize}

\clearpage
\appendix
\section{Commandes et sorties}

\subsection*{Initialisation (cluster)}
\begin{verbatim}
make up-all
\end{verbatim}
\textbf{Sortie (extrait)} :
\begin{verbatim}
[+] Running 3/3
 OK Network apm-data_default   Created
 OK Container apm-data-db-1    Started
 OK Container apm-app-web-1    Started
...
\end{verbatim}

\subsection*{Seed de donnees}
\begin{verbatim}
make seed
\end{verbatim}
\textbf{Sortie (extrait)} :
\begin{verbatim}
Seeded via ORM: inserted=1000
\end{verbatim}

\subsection*{Backups pgBackRest}
\begin{verbatim}
make pgbackrest-full
\end{verbatim}
\textbf{Sortie (extrait)} :
\begin{verbatim}
P00   INFO: backup start
P00   INFO: backup stop
P00   INFO: backup complete
\end{verbatim}

\subsection*{Healthcheck API}
\begin{verbatim}
curl -kI https://localhost:8443/api/health/ | head -n 5
\end{verbatim}
\textbf{Sortie (extrait)} :
\begin{verbatim}
HTTP/2 200
content-type: application/json
...
\end{verbatim}

\section{Structure du projet (extrait)}
\begin{verbatim}
docs/
  report_latex/
    main.tex
    intro.tex
    Chap1_Base.tex
    Chap2_Mise.tex
    Chap3_Pratique.tex
    Chap4_Graphana.tex
    conclusion.tex
    annexes.tex
    references.bib
    images/
docker/
  cluster/
  monitoring/
observability/
\end{verbatim}

\section{Exemples de requetes SQL}
\begin{verbatim}
-- Hits par heure
SELECT bucket AS time, hits
FROM apirequest_hourly
WHERE $__timeFilter(bucket)
ORDER BY bucket;

-- Taux d'erreur (%)
SELECT
  bucket AS time,
  100.0 * errors / NULLIF(hits, 0) AS error_rate
FROM apirequest_hourly
WHERE $__timeFilter(bucket)
ORDER BY bucket;
\end{verbatim}

\section{Extraits de configuration}
\subsection*{Grafana datasource (cluster)}
\begin{verbatim}
type: postgres
url: ${DATA_NODE_IP}:${CLUSTER_DATA_DB_HOST_PORT}
database: ${POSTGRES_DB}
user: ${POSTGRES_READONLY_USER}
sslmode: ${DB_SSLMODE}
\end{verbatim}

\subsection*{pgBackRest (client)}
\begin{verbatim}
repo1-type=s3
repo1-s3-endpoint=https://minio:9000
repo1-s3-bucket=pgbackrest
repo2-s3-bucket=pgbackrest-cold
\end{verbatim}

\section{Extraits de code et resultats}
\subsection*{Ingestion API (Django)}
\begin{lstlisting}[language=Python]
def _parse_ingest_payload(self, data: Any) -> list[Any]:
    if isinstance(data, list):
        return data
    if isinstance(data, dict):
        if "events" not in data:
            raise ValidationError(
                {"detail": "Expected a list payload or an object with an 'events' list."}
            )
        events = data.get("events")
        if not isinstance(events, list):
            raise ValidationError({"events": "Must be a list of event objects."})
        return events
    raise ValidationError({"detail": "Expected JSON list or object payload."})

@action(detail=False, methods=["post"], url_path="ingest")
def ingest(self, request, *args, **kwargs):
    strict = self._get_bool_qp(request, "strict", default=False)
    events = self._parse_ingest_payload(request.data)
    # ...
    if strict and invalid_found:
        return Response(
            {"detail": "Strict mode enabled: payload contains invalid items.",
             "inserted": 0, "rejected": len(events), "errors": errors},
            status=status.HTTP_400_BAD_REQUEST,
        )
    instances = [ApiRequest(**row) for row in validated_rows]
    if instances:
        with transaction.atomic():
            ApiRequest.objects.bulk_create(instances, batch_size=batch_size)
\end{lstlisting}

\subsection*{Migration TimescaleDB (hypertable)}
\begin{lstlisting}[language=Python]
def forwards(apps, schema_editor):
    statements = [
        """
        DO $$
        BEGIN
            CREATE EXTENSION IF NOT EXISTS timescaledb;
        END $$;
        """,
        """
        DO $$
        BEGIN
            PERFORM create_hypertable(
                'observability_apirequest',
                'time',
                if_not_exists => TRUE,
                migrate_data => TRUE,
                create_default_indexes => FALSE,
                chunk_time_interval => INTERVAL '1 day'
            );
        END $$;
        """,
    ]
    with schema_editor.connection.cursor() as cursor:
        for sql in statements:
            cursor.execute(sql)
\end{lstlisting}

\subsection*{Extrait JUnit (Step 1)}
\begin{lstlisting}[language=XML]
<?xml version="1.0" encoding="UTF-8"?>
<testsuites name="APM Observability - Step 1 (CRUD + Filters)" tests="14" time="3.394">
  <testsuite name="00 - Create sample A (older time, 500)" tests="1" failures="0" errors="0">
    <testcase name="Status 201" time="0.252"/>
  </testsuite>
  <testsuite name="10 - GET list (default ordering -time)" tests="3" failures="0" errors="0">
    <testcase name="Status 200" time="0.053"/>
    <testcase name="Has array" time="0.053"/>
    <testcase name="Ordered by -time (non-increasing)" time="0.053"/>
  </testsuite>
</testsuites>
\end{lstlisting}

\subsection*{Resultat KPI (JSON)}
\begin{lstlisting}[language=Java]
{
  "hits": 7,
  "errors": 1,
  "error_rate": 0.14285714285714285,
  "avg_latency_ms": 151.14285714285714,
  "p95_latency_ms": 446.99999999999994,
  "max_latency_ms": 540,
  "source": "hourly"
}
\end{lstlisting}

\subsection*{Top endpoints (JSON)}
\begin{lstlisting}[language=Java]
{
  "source": "hourly",
  "results": [
    {"service": "api", "endpoint": "/health", "hits": 2, "errors": 0, "error_rate": 0.0, "avg_latency_ms": 15.0, "p95_latency_ms": null, "max_latency_ms": 15},
    {"service": "billing", "endpoint": "/x", "hits": 2, "errors": 1, "error_rate": 0.5, "avg_latency_ms": 330.0, "p95_latency_ms": null, "max_latency_ms": 540},
    {"service": "api", "endpoint": "/orders", "hits": 1, "errors": 0, "error_rate": 0.0, "avg_latency_ms": 105.0, "p95_latency_ms": null, "max_latency_ms": 105},
    {"service": "auth", "endpoint": "/login", "hits": 1, "errors": 0, "error_rate": 0.0, "avg_latency_ms": 230.0, "p95_latency_ms": null, "max_latency_ms": 230},
    {"service": "web", "endpoint": "/home", "hits": 1, "errors": 0, "error_rate": 0.0, "avg_latency_ms": 33.0, "p95_latency_ms": null, "max_latency_ms": 33}
  ]
}
\end{lstlisting}

\section{Autres recommandations}
\begin{itemize}
    \item Ajouter une liste d'acronymes (APM, CAGG, WAL, MVCC).
    \item Documenter les prerequis exacts (versions Docker/Python/Node).
    \item Ajouter des captions avec sources pour les figures.
    \item Normaliser la terminologie (FR/EN) dans tout le document.
    \item Ajouter un court paragraphe de limites et risques.
\end{itemize}


\clearpage
\bibliographystyle{IEEEtran}
\bibliography{references}
\end{document}

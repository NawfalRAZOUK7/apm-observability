\section{Application pratique}

\subsection{Sauvegardes et restauration}

\subsubsection*{Backups hot et cold}
Les scripts de sauvegarde supportent les backups à chaud et à froid. Toutes les opérations sont automatisées et idempotentes.

\subsubsection*{Stockage compatible S3}
Les backups peuvent être exportés vers un stockage compatible S3, facilitant la conservation et la restauration des données.

\subsection{Données de test}

\subsubsection*{Seeding de la base avec Faker}
Pour les tests et démonstrations, la base peut être peuplée avec des données factices générées via la librairie Faker.

\subsection{Tolérance aux pannes}

\subsubsection*{Simulation de pannes}
Des tests de tolérance sont possibles en simulant l’arrêt de nœuds Docker pour vérifier la résilience du cluster.

\subsubsection*{Procédures de recovery}
La restauration automatique depuis les backups permet de remettre rapidement le système en état de fonctionnement.

\subsection{Optimisation}

\subsubsection*{Index et performances}
L’utilisation des hypertables, des agrégats continus et d’index spécifiques permet d’optimiser les performances des requêtes analytiques, notamment pour les KPI et les tableaux de bord.


